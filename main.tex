\documentclass[12pt]{article}

\usepackage{colourmath}

\newcommand\boldblue[1]{\textcolor{blue}{\textbf{#1}}}
\newcommand\boldred[1]{\textcolor{red}{\textbf{#1}}}
\newcommand\boldgreen[1]{\textcolor{green}{\textbf{#1}}}

\title{Positional Games}
\author{Kok Ying Hao}
\date{\today}

\setlength\parindent{0pt}
\setlength\parskip{2ex}
\allowdisplaybreaks

\begin{document}

\maketitle

\noindent An extremely far-reaching generalisation of the simple game of Tic-Tac-Toe takes the following
form. We have an arbitrary finite set as the board of the game, and some of its subsets are
designated as winning sets. Two players alternately play on an unoccupied cell of the board,
and the first player to complete a winning set is the winner — otherwise, the game ends in a
draw. This is called the strong positional game (or strong game in short).

In the weak version, also called the Maker-Breaker version, the second player’s aim is not to
occupy a winning set but just to prevent the first player from doing so. For both versions,
the interest is in the general results about the games. One may also focus on the more natural
multidimensional generalisation of Tic-Tac-Toe: the Hales-Jewett game. Very roughly speaking,
a fair amount is known for Maker-Breakers while almost nothing is known for strong games.

The aim of the course is to provide a brief introduction to this combinatorial discipline to those
with interest and basic knowledge in combinatorics. We will be covering some very elegant
results in the field, as well as many of its standing challenges and open problems. Among
others, these include the Strategy Stealing argument, the Pairing Strategies, the Erd\"{o}s-Selfridge
Theorem, the Pairing Conjecture and the Neighbourhood Conjecture.

\tableofcontents

% \clearpage



\newpage
\subfile{section/sect01}


\end{document}